

\chapter{图}

\[\text{图}\begin{cases}
\text{定义} \\ 
\text{图结构的存储}\begin{cases}
\text{邻接矩阵法,邻接表法} \\ 
\text{邻接多重表,十字链表}\end{cases} \\ 
\text{图的遍历}\begin{cases}
\text{深度优先} \\ 
\text{广度优先}\end{cases} \\ 
\text{相关应用}\begin{cases}
\text{最小生成树:Prim算法、Kruskal算法} \\ 
\text{最短路径:Dijkstra算法、Floyd算法} \\ 
\text{拓扑排序:AOV网} \\ 
\text{关键路径:AOE网}\end{cases}
\end{cases}\]

\section{定义}

\subsubsection{有向图}

\subsubsection{无向图}

\subsubsection{简单图}

()不存在重复边;

()不存在顶点到自身的边;



\subsubsection{多重图}

()某两个顶点之间边数大于1;

()允许顶点通过一条边和自身关联;



\subsubsection{顶点的度}
依附顶点的边的条数记为\(TD(v)\);

无向图边数的二倍等于各顶点度数的总和;


\subsubsection{入度出度}
有向图中分为入度出度;

顶点的度等于其入度和出度之和;

有向图全部顶点的入度之和与出度之和相等,且等于边数;


\subsubsection{路径}
顶点到顶点之间的一条路径是指顶点序列;

由顶点和相邻顶点序偶构成的边所形成的序列;

\paragraph{路径长度}
路径上边的数量称为路径长度;

\paragraph{环}
第一个顶点和最后一个顶点相同的路径称为回路或环;

若一个图有n个顶点且有大于n - 1条边,则一定有环;

\subparagraph{判断}
对有向图,使用深度优先遍历或拓扑排序算法;


\subsubsection{简单路径}

路径序列中,顶点不重复出现的路径为简单路径;

除第一个顶点和最后一个顶点外不重复出现的回路为简单回路;


\subsubsection{距离}

若两顶点间存在最短路径,其长度为距离;若不存在路径则距离为无穷;


\subsubsection{子图}
设两个图\(G = (V, E), G' = (V', E')\),若\(V'\)是\(V\)的子集,\(E'\)是\(E\)的子集,则\(G'\)是\(G\)的子图;

若\(E'\)中边对应顶点不是\(V'\)的元素,则无法构成图;

若存在满足\(V(G') = V(G)\)的子图\(G'\),则其为\(G\)的生成子图;


\subsubsection{连通}

\paragraph{连通分量}
无向图中的极大连通子图为连通分量

\subparagraph{性质}
可能存在环;

\paragraph{极大连通子图}
无向图中未连通的子图;


\subsubsection{强连通图}

在有向图中,若一对顶点相互连通,则这对顶点强连通;

若图中任意一对顶点强连通,则为强连通图;

\paragraph{强连通分量}
有向图的极大强连通子图为强连通分量


\subsubsection{生成树}
连通图的生成树是包含图中全部顶点的极小连通子图。若图中顶点数为n,则生成树含n - 1条边;
\subparagraph{性质}
无环


\subsubsection{边的权}


\subsubsection{完全图}

有\(\dfrac{n(n - 1)}{2}\)条边的无向图为完全图;

有\(n(n - 1)\)条边的有向图为有向完全图;


\subsubsection{有向树}
一个顶点入度为0,其余顶点入度为1的有向图为有向树;


\subsubsection{拓扑排序}
适用于有向无环图DAG,拓扑排序是DAG的所有顶点的线性序列,满足
\begin{itemize}
    \item 每个顶点出现且只出现一次
    \item 若存在路径A->B,则序列中A在B前
\end{itemize}



\section{存储}

\subsubsection{邻接矩阵法}

\[A[i][j] = \begin{cases}
    1, \text{是边} \\ 
    0, \text{不是边}
\end{cases}\]

带权图\[A[i][j] = \begin{cases}
    w_{ij}, \text{是边} \\ 
    0 || \infty, \text{不是边}
\end{cases}\]

\paragraph{无向图邻接矩阵}
一定对称且唯一,仅存储上下三角矩阵元素

\paragraph{度的计算}
\subparagraph{对无向图}
邻接矩阵的第i行(或第i列)非零元素(或非\(\infty\)元素)的个数正好是顶点i的度TD(\(v_i\))

\subparagraph{对有向图}
邻接矩阵的第i行非零元素(或非\(\infty\)元素)的个数正好是顶点i的出度OD(\(v_i\)),第i列非零元素(或非\(\infty\)元素)的个数正好是顶点i的入度ID(\(v_i\))

\paragraph{矩阵的幂次方}
设图G的邻接矩阵为A,则\(A^n\)的元素\(a^n_{i, j}\)为从顶点\(i\)导顶点\(j\)的长度为n的路径的数量


\subsubsection{邻接表法}
对图G中的每一个顶点\(v_i\)建立一个单链表,第i个单链表中的结点表示依附于顶点\(v_i\)的边,该单链表为顶点\(v_i\)的边表(对有向图为出边表)


\paragraph{存储空间复杂度}
若\(G\)无向图,则所需存储空间为 \(O(|V| + 2|E|)\);若\(G\)为有向图,则所需的存储空间为\(O(|V| + |E|)\);

\paragraph{度的计算}
无向图邻接表中,求顶点的度只需计算其邻接表中的边表节点个数;

有向图的邻接表中,求顶点的出度只需计算其邻接表中的边表节点个数,入度需遍历全部邻接表;

\paragraph{图的邻接表表示不唯一}

\paragraph{建立时间复杂度}
建立邻接表需要遍历所有顶点和边
\subparagraph{对无向图}
设n个顶点e条边,需\(n + 2e\)次操作,时间复杂度\(O(n + e)\)

\paragraph{删除时间复杂度}
\subparagraph{对有向图}
删出边\(O(n)\),删入边\(O(n + e)\),总时间复杂度\(O(n + e)\)



\subsubsection{十字链表}
有向图的链式存储结构;每条弧用一个节点表示,每个顶点用一个节点表示;

弧节点有五个域:tailvex \(\&\) headvex分别存放弧尾与弧首两个顶点的编号;hlink指向弧首相同的下一条弧;tlink指向弧尾相同的下一条弧;info存放弧的相关信息。

顶点节点有三个域:data存放数据信息;firstin指向以该节点为弧首的第一条弧;firstout指向以该顶点为弧尾的第一条弧;

图的十字链表表示不唯一;一个十字链表表示唯一确定一个图;


\subsubsection{邻接多重表}
无向图的链式存储结构;

每条边用一个结点表示,其中,ivex\(\&\)jvex存放该边依附的两个顶点的编号;ilink指向依附于顶点ivex的下一条边;jlink指向依附于顶点jvex的下一条边;info存放该边相关信息;

每个顶点用一个结点表示,其中,data存放相关信息;firstedge指向依附于该顶点的第一条边;

所有依附于同一顶点的边串联在同一链表中,

就无向图而言,同一边在邻接表中用两个结点表示,而在邻接多重表中只有一个结点;


\subsubsection{四种方式比较}

\begin{center}
\begin{tabular}{c|c|c|c|c}
\hline
& \text{邻接矩阵} & \text{邻接表} & \text{十字链表} & \text{邻接多重表} \\ 
\hline
\multirow{2}{*}{空间复杂度} & \multirow{2}{*}{\(O(|V|^2)\)} & 无向图:\(O(|V| + 2|E|)\) & \multirow{2}{*}{\(O(|V| + |E|)\)} & \multirow{2}{*}{\(O(|V| + |E|)\)} \\ 
& & 有向图:\(O(|V| + |E|)\) & & \\ 
\hline
\multirow{2}{*}{找相邻边} & 遍历对应行或列的 & 找有向图的入度必须 & \multirow{2}{*}{很方便} & \multirow{2}{*}{很方便} \\ 
& 时间复杂度为\(O(|V|)\) & 遍历整个邻接表 & & \\ 
\hline
\multirow{2}{*}{删除边或顶点} & 删除边方便,删除顶点 & 无向图删除边 & \multirow{2}{*}{很方便} & \multirow{2}{*}{很方便} \\ 
& 需要大量移动数据 & 或顶点都不方便 & & \\ 
\hline
适用于 & 稠密图 & 稀疏图 & 只存有向图 & 只存无向图 \\ 
\hline
表示方法 & 唯一 & 不唯一 & 不唯一 & 不唯一 \\ 
\hline
\end{tabular}
\end{center}


\subsection{图的基本操作}
基本操作独立于存储结构;对于不同的存储方式,操作的实现有不同的性能;

操作包括\begin{itemize}
    \item Adjacent(G, x, y):判断图G是否存在边(x, y)
    \item Neighbors(G, x):列出图G中与结点x邻接的边
    \item InsertVertex(G, x):在图G中插入顶点x
    \item DeleteVertex(G, x):从图G中删除顶点x
    \item AddEdge(G, x, y):...
    \item ...
\end{itemize}

\section{遍历}

\subsubsection{BFS}
广度优先搜索,类似树的层次遍历,辅助数组visited[]标志顶点是否被访问过

队列实现;

\paragraph{性能}
最坏空间复杂度\(O(|V|)\)

\subparagraph{邻接表}
搜索顶点时间复杂度\(O(|V|)\),搜索边时间复杂度\(O(|E|)\),总时间复杂度\(O(|V| + |E|)\)

\subparagraph{邻接矩阵}
搜索每个顶点邻接点时间\(O(|V|)\),总时间\(O(|V|^2)\)

\paragraph{单源最短路径}
设图\(G = (V, E)\)为非带权图,定义从顶点u到顶点v的最短路径\(d(u, v)\)为u到v的任何路径中最少的边数;若u到v没有通路,则\(d(u, v) = \infty\)

使用BFS求解满足上述定义的非带权图的单源最短路径问题;

\paragraph{广度优先生成树}
广度遍历中,可得到一颗遍历树;

图的邻接矩阵唯一,其广度优先生成树唯一;

图的邻接表不唯一,其广度优先生成树不唯一;


\subsubsection{DFS}
深度优先搜索,类似树的先序遍历

性质是顶点所有后继顶点出栈后才出栈。

\paragraph{性能}
需递归工作栈,空间复杂度\(O(|V|)\)

\subparagraph{邻接表}
总时间复杂度\(O(|V| + |E|)\)

\subparagraph{邻接矩阵}
总时间\(O(|V|^2)\)

\paragraph{深度优先生成树}
连通图产生深度优先生成树,非连通图产生深度优先生成森林

\paragraph{可判断是否有环}


\subsubsection{遍历与连通性}

\paragraph{对无向图}
若连通,则一次遍历即可访问所有顶点;反之一次遍历只能访问该顶点所在连通分量的所有顶点;

\paragraph{对有向图}
若从顶点到每个顶点都有路径,则可访问所有顶点;

连通的有向图的非强连通分量一次遍历不一定能访问所有顶点;

\section{应用}
\subsection{最小生成树}
连通图的生成树包含图的所有顶点,且只含尽可能少的边。生成树少一条边则非连通,多一条边则有回路。

对一个带权连通无向图G,生成树不同,树的权可能不同;权值之和最小的生成树为最小生成树。

\paragraph{性质}
\begin{itemize}
    \item 若图G中存在权值相同的边,则G的最小生成树可能不唯一,树形可能不唯一;各边权值相同时最小生成树唯一;
    \item 若无向连通图边数比顶点数少1,则最小生成树是其本身;
    \item 最小生成树权值和唯一;
    \item 最小生成树边数为顶点数 - 1;
    \item 任意两顶点之间路径不一定是最短路径;
    \item 设\(G = (V, E)\)是带权连通无向图,U是顶点集V的非空子集,若(u, v)为一条具有最小权值的边,其中\(u \in U, v \in V - U\),则必存在一颗包含边\((u, v)\)的最小生成树;
\end{itemize}

\subsubsection{Prim算法}

\paragraph{时间复杂度}
\(O(|V|^2)\),与边\(|E|\)无关,适用于边稠密图求最小生成树;

\subsubsection{Kruskal算法}
通常使用堆存放边的集合

\paragraph{时间复杂度}
每次选择最小权值的边\(O(\log_2|E|)\),每次使用并查集判断两个顶点是否属于同一集合\(O(\alpha(|V|))\),\(\alpha(|V|)\)增长极其缓慢,可视为常数。总时间\(O(|E|\log_2|E|)\),不依赖于|V|,适用于边稀疏顶点多的图。


\subsection{最短路径}
最短路径一定是简单路径

\subsubsection{Dijkstra算法}
求某点到其它各顶点的最短路径;基于贪心策略;
不适用于边带负权值;

\paragraph{时间复杂度}
\(O(|V|^2)\),求源点到某特定点的最短路径\(O(|V|^2)\)

\subsubsection{Floyd算法}
求每对顶点间的最短路径

\paragraph{时间复杂度}
\(O(|V|^3)\)

\subsubsection{算法比较}
\begin{center}
    \begin{tabular}{c | c | c | c}
         & BFS & Dijkstra & Floyd \\
        \hline
        用途 & 求单源最短路径 & 求单源最短路径 & 求各顶点之间的最短路径 \\ 
        \hline
        无权图 & 适用 & 适用 & 适用 \\ 
        \hline
        带权图 & 不适用 & 适用 & 适用 \\ 
        \hline
        带负权值图 & 不适用 & 不适用 & 适用 \\ 
        \hline
        带负权值回路图 & 不适用 & 不适用 & 不适用 \\ 
        \hline
        时间复杂度 & \(O(|V|^2)\) 或\(O(|V| + |E|)\) & \(O(|V|^2)\) & \(O(|V|^3)\)
    \end{tabular}
\end{center}

\subsection{有向无环图描述表达式}
构建表达式的有向无环图
\begin{enumerate}
    \item 转化为有向二叉树
    \item 将有向二叉树中相同的项合并去重,得到有向无环图
\end{enumerate}


\subsection{拓扑排序}
\paragraph{AOV网}
若有向无环图表示一个工程,其顶点表示活动,用有向边\(<V_i, V_j>\)表示活动\(V_i\)必须先于活动\(V_j\)的关系,则将这种有向图称为顶点表示活动的网络,简称AOV网。

AOV网中,活动\(V_i\)是活动\(V_j\)的直接前驱,\(V_j\)是\(V_i\)的直接后继,具有传递性,且不能以自身为前期或后继。

\paragraph{定义}
图论中,由一个有向无环图的顶点组成的序列,当且仅当满足下列条件时称为该图的一个拓扑排序:
\begin{itemize}
    \item 每个顶点出现且仅出现一次;
    \item 若顶点A在序列中排在顶点B前,则图中不存在从B到A的路径;
\end{itemize}
拓扑排序是对有向无环图顶点的一种排序,若存在一条从顶点A到顶点B的路径,则排序中B在A后。

每个AOV网有一个或多个拓扑排序序列。

\subsubsection{与回路关系}

\begin{enumerate}
    \item 从AOV网中选择一个无前驱的顶点并输出;
    \item 从网中删除该点及所有以其为起点的有向边;
    \item 重复1,2直到AOV网为空或不存在无前驱的顶点,后者说明图有环;
\end{enumerate}

\subsubsection{效率}

\subparagraph{邻接表}
时间复杂度\(O(|V| + |E|)\)

\subparagraph{邻接矩阵}
时间复杂度\(O(|V|^2)\)

\subsubsection{DFS实现拓扑排序}

\subsubsection{逆拓扑排序}
\begin{enumerate}
    \item 从AOV网中选择一个无后继的顶点并输出;
    \item 从网中删除该点及所有以其为终点的有向边;
    \item 重复1,2直到AOV网为空;
\end{enumerate}


\subsubsection{性质}

\begin{itemize}
    \item 顶点数大于1的强连通图不能进行拓扑排序
    \item 若拓扑序列唯一,则图中入度为0和出度为0的点都仅有1个
    \item 暂存入度为0的点,可以用栈和队列
\end{itemize}

\paragraph{存在性与唯一性}
\begin{itemize}
    \item 各顶点为线性序列是拓扑序列唯一的充分非必有条件
    \item 若每次输出顶点时,入度为0的顶点唯一,则拓扑序列唯一
\end{itemize}


\subsection{关键路径}

\subsubsection{AOE网}
带权有向图中,以顶点表示事件,有向边表示活动,边上权值表示完成活动的开销,称之为用边表示活动的网络,AOE网。

\paragraph{性质}
\begin{itemize}
    \item 只有顶点所代表事件发生后,从该顶点出发的各有向边所代表的活动才能开始;
    \item 只有进入该顶点的各有向边所代表的活动都已结束时,该顶点所代表的事件才能发生;
    \item 一个事件的最早发生时间与以该事件为始的弧的活动的最早开始时间相同;
\end{itemize}


\paragraph{事件发生时间}

\subparagraph{最早}

\subparagraph{最迟}
\(\min\){以该事件为尾的弧的活动的最初开始时间}或\(\min\){以该事件为尾的弧所指事件的最迟发生时间与该弧的活动的持续时间之差}


\paragraph{与AOV网区别}
\begin{itemize}
    \item 都是有向无环图
    \item AOE网边有权值,AOV网边无权值
\end{itemize}


\subsubsection{定义}
AOE网中,从源点到汇点的所有路径中,具有最大路径长度的路径为\textit{关键路径},关键路径上的活动为\textit{关键活动};

完成整个工程的最短时间是关键路径的长度,即各关键活动的权值总和。

\subsubsection{求解步骤}
\begin{enumerate}
    \item 从源点出发,令\(v_e(\text{源点}) = 0\),按拓扑有序求其其余顶点的最早发生时间\(v_e()\)
    \item 从汇点出发,令\(v_l(\text{汇点}) = v_e(\text{汇点})\),按拓扑有序求其余顶点的最迟发生时间\(v_l()\)
    \item 根据各顶点的\(v_e()\)值求所有弧的最早开始时间\(e()\)
    \item 根据各顶点的\(v_l()\)值求所有弧的最迟开始时间\(l()\)
    \item 求AOE网中所有活动的差额\(d()\),找出所有\(d() = 0\)的活动构成关键路径
\end{enumerate}


\subsection{时间复杂度比较}
\begin{center}
    \begin{tabular}{c|c|c|c|c|c|c|c|c}
         & Dijkstra & Floyd & Prim & Kruskal & DFS & BFS & 拓扑排序 & 关键路径 \\ 
        \hline
        邻接矩阵 & \(O(n^2)\) & \(O(n^3)\) & \(O(n^2)\) & & \(O(n ^2)\) & \(O(n^2)\) & \(O(n^2)\) & \(O(n^2)\) \\ 
        \hline
        邻接表 & & & & \(O(e\log_2e)\) & \(O(n + e)\) & \(O(n + e)\) & \(O(n + e)\) & \(O(n + e)\)
    \end{tabular}
\end{center}



