
\chapter{排序}

\section{概念}

\subsection{排序}
重新排列表中元素,使元素满足按关键字有序的过程。
\begin{enumerate}
    \item 输入:n个记录\(R_1, ..., R_n\),对应关键字为\(k_1, ..., k_n\)
    \item 输出:输入序列的一个重排\(R'_1, ..., R'_n\),使得\(k'_1 <= ... <= k'_n\),比较符号任意。
\end{enumerate}


\subsection{算法稳定性}
若待排序表中有两个元素关键字相同,排序前后两个元素前后位置没有改变,则该排序算法是稳定的。


\subsection{算法类别}
根据数据元素是否完全放在内存中,排序算法分为两类:
\begin{enumerate}
    \item 内部排序\begin{itemize}
        \item 插入排序
        \item 选择排序
        \item 冒泡排序
    \end{itemize}
    \item 外部排序\begin{itemize}
        \item 拓扑排序
    \end{itemize}
\end{enumerate}


\section{插入排序}

\subsection{直接插入排序}

\subsubsection{效率}

\paragraph{空间}
\(O(1)\)

\paragraph{时间}
\begin{itemize}
    \item 最好情况下,元素已有序,复杂度\(O(n)\);
    \item 最坏情况下,表中元素顺序与结果顺序相反,复杂度\(O(n^2)\)
    \item 平均情况下,取最好情况与最坏情况的平均值,总的比较次数与移动次数均约为\(n^2 / 4\),复杂度\(O(n^2)\)
\end{itemize}


\subsubsection{适用于}
顺序存储和链式存储的线性表,采用链式存储无需移动元素


\subsection{折半插入排序}

\subsubsection{效率}

\paragraph{时间}
比较次数与初始状态无关,复杂度约\(O(n\log_2n)\);但移动次数未改变,依赖于初始状态。综上,时间复杂度为\(O(n^2)\)


\subsubsection{适用于}
顺序存储的线性表,不稳定


\subsection{希尔排序}
也称缩小增量排序。

\subsubsection{思想}
先将待排表分为若干形如\([i, i + d, i + 2d, ..., i + kd]\)的特殊子表,即把相隔某个增量的记录组成一个子表,对各个子表进行直接插入排序,当整个表基本有序时,对全体进行一次直接插入排序。

\subsubsection{效率}

\paragraph{空间}
\(O(1)\)

\paragraph{时间}
时间复杂度依赖于增量序列的函数,涉及数学上仍未解决的难题。当n在某个特定范围时,时间复杂度约为\(O(n^{1.3})\);最坏情况下复杂度为\(O(n^2)\)


\subsubsection{适用于}
顺序存储的线性表


\section{交换排序}

\subsection{冒泡}
稳定

\subsubsection{效率}
\paragraph{空间}
\(O(1)\)

\paragraph{时间}

最坏情况时间复杂度\(O(n^2)\),平均时间复杂度\(O(n^2)\)


\subsubsection{适用于}
顺序存储和链式存储的线性表


\subsection{快速}
不稳定。

\subsubsection{思想}
在待排表中任取一个元素pivot作为枢轴(或基准),通常取首元素,通过一趟排序将待排表分为两部分,左部分中所有元素小于pivot,右部分中所有元素大于pivot。这个过程称为一次划分。分别递归地对两个子表重复上述过程,直至每个部分中只有一个元素或为空位置。
\begin{enumerate}
    \item 设两个指针i,j。i指向数组首,j指向数组尾。通常以首元素为枢轴,则从j开始。
    \item j向前找到第一个小于枢轴的元素,将该元素移动到i所在位置;
    \item i向后找到第一个大于枢轴的元素,将该元素移动到j所在位置;
    \item 重复2、3,直到i = j。
\end{enumerate}


\subsubsection{效率}

\subparagraph{空间}
最好情况\(O(\log_2n)\)最坏情况下,进行\(n - 1\)次递归调用,栈深度\(O(n)\),平均情况\(O(\log_2n)\)。递归次数与分区处理顺序无关。

\subparagraph{时间}
最坏情况\(O(n^2)\)


\subsubsection{适用于}
顺序存储的线性表。


\subsubsection{结论}
\begin{itemize}
    \item 对n个元素进行第一趟快排后,会确定一个基准元素,根据这个基准元素的位置有两种情况:\begin{enumerate}
    \item 基准元素在首段或尾端。对剩下n - 1个元素构成的子序列进行第二趟快排,再确定一个基准元素。两趟至少确定两个基准元素的最终位置,其中至少一个基准元素在首端或尾端。
    \item 基准元素不在首端或尾端。第二趟对两个子序列分别划分,各确定一个基准元素,两趟至少确定三个元素最终位置。
    \end{enumerate}
    \item 使用快速排序处理时,当表本身已有序或逆序时,速度最慢;当每趟枢轴值都把表等分为长度相近的两个子表时,速度最快。
\end{itemize}


\section{选择排序}

\subsection{简单选择排序}
不稳定。

\subsubsection{效率}

\paragraph{空间}
\(O(1)\)

\paragraph{时间}
移动次数最好0次,最坏不超过\(3(n - 1)\)次;比较次数始终\(n(n - 1) / 2\)次;时间复杂度始终\(O(n^2)\)。


\subsubsection{适用于}
顺序存储和链式存储的线性表,及关键字较少的情况。


\subsection{堆排序}

\subsubsection{堆}
见\ref{堆}。

\subsubsection{思想}
\begin{enumerate}
    \item 以大根堆为例。先建成初始堆,由于堆本身特点,堆顶元素为最大值;
    \item 输出堆顶元素,通常将堆底元素送入堆顶此时根结点不满足大根堆的性质;
    \item 堆被破坏,堆顶元素向下调整使其保持大顶堆性质,再输出堆顶元素;
    \item 重复步骤,直到堆中仅一个元素。
\end{enumerate}


\section{归并排序}
稳定。

\subsection{性能}
共需\(\log_2n\)趟。

空间:\(O(n)\)

时间:每趟\(O(n)\),共\(O(n\log_2n)\)


\subsection{适用于}
顺序存储和链式存储的线性表。


\section{基数排序}
稳定。不基于比较和移动。

\subsection{性能}
空间:需要r个队列,故\(O(r)\)

时间:一趟分配\(O(n)\),一趟收集\(O(r)\),共\(O(d(n + r))\)


\subsection{适用于}
顺序存储和链式存储的线性表。


\section{计数排序}
稳定。不基于比较。

\subsection{适用于}
顺序存储的线性表,元素为整数且范围不大。


\section{内部排序算法}




