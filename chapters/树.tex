
\chapter{树}
n个结点的树有n - 1条边;


\section{二叉树}

二叉树是逻辑结构。

先序序列与中序序列的关系相当于以先序序列为入栈次序,中序序列为出栈次序。


\section{特殊二叉树}

\subsection{满二叉树}


\subsection{完全二叉树}


\subsection{二叉排序树}
见\ref{二叉排序树}

\subsection{平衡二叉树}
见\ref{平衡二叉树}

\subsection{正则二叉树}


\section{二叉树遍历}

\subsection{先序}

\subsection{中序}

\subsection{后序}

\subsection{层次}

\subsection{由遍历序列构造二叉树}


\section{线索二叉树}

线索二叉树是二叉树加上线索的链表结构,是计算机内的存储结构,是物理结构。

二叉树线索化后无法解决后续线索二叉树求后序后继。

后续线索二叉树遍历仍需栈的支持。

\subsection{中序线索二叉树}
即按\textbf{中序遍历}顺序连接。

顺序第一个左指针指向空,最后一个右指针指向空。中序遍历第一个必无左孩子,最后一个必无右孩子。


\subsection{先序后序}


\section{哈夫曼树}

\paragraph{带权路径长度WPL}
 = \(\sum\)节点路径长度 * 权值

\paragraph{加权平均长度}
 = \(\dfrac{\text{WPL}}{\text{权值和}}\)

\subsection{前缀编码}

\subsection{计算}

\paragraph{1}
设哈夫曼树度为m,叶结点数为n,求非叶结点个数;

设度为m的结点有\(n_m\)个,度为0的结点为n个,则结点总数\(N = n_m + n\),有N个结点的哈夫曼数有N - 1条分支,则\(mn_m = N - 1 = n_m + n - 1\),则\(n_m = \dfrac{n - 1}{m - 1}\)


\subsubsection{123}


\section{并查集}

是用双亲表示法存储的树;

可以检测图中是否存在环路问题;

可以用于判断无向图的联通性;

长度为n的并查集中进行查找操作的时间复杂度为\(O(\log_2n)\)

\subsection{Union}

可根据集合规模将小集合合并到大集合中;


\section{堆}\label{堆}
n个关键字序列L[1, ..., n]称为堆,当且仅当该序列满足:
\begin{enumerate}
    \item L(i) >= L(2i) 且 L(i) >= L(2i + i)\ \ 或
    \item L(i) <= L(2i) 且 L(i) <= L(2i + 1)\ \ \((1 <= i <= \lfloor n / 2\rfloor\)
\end{enumerate}

可将堆视为一棵完全二叉树。满足条件1的堆称为大根堆(大顶堆),大根堆的最大元素放在根结点,其任意一个非根结点的值小于等于其双亲结点值。满足条件2的堆为小根堆(小顶堆),小根堆的定义相反。

\paragraph{堆的应用}
堆排序和优先队列。

\paragraph{建堆}




