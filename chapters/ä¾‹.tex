
\chapter{例}

\section{树}

\subsubsection{树结点数}
在一棵度数为4的树T中,若有20个度4的结点,10个度3的结点,1个度2的结点,10个度1的结点,则T的叶结点数为?

\subparagraph{解}
设树中度为i(i=0,1,2,3,4)的结点数分别为Ni,树中结点总数为 N,则树中各结点的度之和等于N-1,即 N = 1+N1+2N2+3N3+4N4 = N0+ N1+N2+N3+N4,根据题设中的数据,即可得到 N0 = 82,即树T的叶结点的个数是82。


\section{图}

\subsubsection{无向图连通条件}
若无向图G=(V, E)中含有7个顶点,要保证图G在任何情况下都是连通的,则需要的边数最少为?

\subparagraph{解}
要保证无向图G在任何情况下都是连通的,即任意变动图G中的边,G始终保持连通,首先需要G的任意六个结点构成完全连通子图G1,需15条边,然后再添一条边将第7个结点与G1连接起来,共需16条边。 


\subsubsection{邻接矩阵有向图拓扑序列}
若用邻接矩阵存储有向图,主对角线以下元素均为0,则该图拓扑序列:存在,可能不唯一。


\subsubsection{有向图拓扑序列}
若一个有向图中的顶点不能排成一个拓扑序列,表明其中存在一个顶点数目大于1的回路,该回路构成一个强连通分量。


