
\chapter{树}
n个结点的树有n - 1条边;


\section{二叉树}

二叉树是逻辑结构。

线索二叉树是二叉树加上线索的链表结构,是计算机内的存储结构,是物理结构。

二叉树线索化后无法解决后续线索二叉树求后序后继。

后续线索二叉树遍历仍需栈的支持。


先序序列与中序序列的关系相当于以先序序列为入栈次序,中序序列为出栈次序。


\section{哈夫曼树}

\paragraph{带权路径长度WPL}
 = \(\sum\)节点路径长度 * 权值

\paragraph{加权平均长度}
 = \(\dfrac{\text{WPL}}{\text{权值和}}\)

\subsection{前缀编码}

\subsection{计算}

\paragraph{1}
设哈夫曼树度为m,叶结点数为n,求非叶结点个数;

设度为m的结点有\(n_m\)个,度为0的结点为n个,则结点总数\(N = n_m + n\),有N个结点的哈夫曼数有N - 1条分支,则\(mn_m = N - 1 = n_m + n - 1\),则\(n_m = \dfrac{n - 1}{m - 1}\)


\subsubsection{123}


\section{并查集}

是用双亲表示法存储的树;

可以检测图中是否存在环路问题;

可以用于判断无向图的联通性;

长度为n的并查集中进行查找操作的时间复杂度为\(O(\log_2n)\)

\subsection{Union}

可根据集合规模将小集合合并到大集合中;

